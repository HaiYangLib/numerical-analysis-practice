\documentclass[a4paper]{article} 

%中文环境设置
\usepackage{xeCJK} 
\usepackage{indentfirst}
\setlength{\parindent}{2em}
\usepackage{enumitem}

\usepackage{abstract}
\renewcommand{\abstractname}{摘要}
\providecommand{\keywords}[1]{\textbf{\textit{关键词}} #1}

\setCJKmainfont{STSong} % 中文主字体设置 

\usepackage[colorlinks,linkcolor=blue, citecolor=blue]{hyperref}

% 常用宏包
\usepackage{float}
\usepackage{stfloats}
\usepackage{graphicx}
\usepackage{color}
\usepackage{supertabular}

% 代码环境设置
\usepackage{listings}
\lstset{
	columns=fullflexible,
 	frame=single,
 	breaklines=true,
}
\definecolor{lightgray}{gray}{0.9}
\newcommand{\inlinecode}[2]{\colorbox{lightgray}{\lstinline[language=#1]$#2$}}

% 页面段落设置
\usepackage{multicol}
\usepackage{geometry}
\geometry{left=3.18cm, right=3.18cm, top=2.54cm, bottom=2.54cm}
\linespread{1.3}
%\setlength{\parskip}{0.5em} 

% 数学环境设置
\usepackage{amsmath}
\usepackage{amssymb}
\usepackage{amsthm}
\usepackage{amsfonts}
\usepackage{mathrsfs}
\newtheorem{myDef}{Definition} 
\newtheorem{myThm}{Theorem}
\newtheorem{myProp}{Property}

\begin{document} 
\title{常微分方程数值解实验题}
\author{吴佳龙 2018013418}
\date{}
\maketitle

\begin{abstract}
	结合理论分析和编程计算,运用不同方法计算了一常微分方程初值问题的数值解,并与精确解比较。运用的方法分别为:古典四级四阶 Runge-Kutta 方法和隐式二级四阶 Runge-Kutta 方法(Gauss 方法)。
\end{abstract}

%\keywords{one, two, three, four}

\begin{multicols}{2}

\begin{section}{问题}

	求解:$$\left\{\begin{array}{l}{\frac{d u}{d t}=-2000 u(t)+999.75 v(t)+1000.25} \\ {\frac{d v}{d t}=u(t)-v(t)}\end{array}\right.$$ 初始条件为 $u(0)=0, v(0)=-2$。其精确解为 $$\left\{\begin{array}{l}{u(t)=-1.499875 e^{-0.5 t}+0.499875 e^{-2000.5 t}+1} \\ {v(t)=-2.99975 e^{-0.5 t}-0.00025 e^{-2000.5 t}+1}\end{array}\right.$$
	
	分别用古典四级四阶 Runge-Kutta 方法和隐式二级四阶 Runge-Kutta 方法计算,计算区间取成 $[0,20]$,并与精确解比较。

\end{section}

\begin{section}{古典四级四阶 Runge-Kutta 方法}
	
	\begin{subsection}{算法原理}
	
		\begin{subsubsection}{用 Taylor 展开构造高阶数值方法}
			
			取 $y(x+h) \approx y(x)+hy^\prime(x)$ 得到单步法,即 1 阶 Euler 方法 $$y_{n+1} = y_{n}+hf(x_n,y_n)$$
			
			取 $y(x+h) \approx y(x)+hy^\prime(x) + {1\over 2}h^2y^{\prime\prime}(x)$ 得到 2 阶单步法 \begin{align}
				\nonumber
				y_{n+1}&=y_{n}+h f\left(x_{n}, y_{n}\right)\\  &+\frac{h^{2}}{2}\left[\frac{\partial f}{\partial x}+f \frac{\partial f}{\partial y}\right]\left(x_{n}, y_{n}\right)
				\label{taylor}
				\end{align}
			
			如此构造下去,可得到三阶方法以及更高阶的方法,但是该类方法需要计算很多偏导数,并不实用。
			
		\end{subsubsection}
		
		\begin{subsubsection}{Runge-Kutta 方法}
		
			Runge-Kutta 方法采用了不同点上函数值的不同组合来提高精度同时避免函数 $f$ 的偏导数的计算。其一般形式为 
			
			$$y_{n+1}=y_{n}+h \varphi\left(x_{n}, y_{n} ; h\right)$$ $$\varphi\left(x_{n}, y_{n} ; h\right)=\sum_{i=1}^{s} b_{i} k_{i}$$ $$k_{i}=f\left(x_{n}+c_{i} h, y_{n}+h \sum_{j=1}^{s} a_{i j} k_{j}\right)$$ $$c_{i}=\sum_{j=1}^{s} a_{i j}, \quad i=1,2, \cdots, s$$
			
			例如,在二级方法中,将 $k_2$ 进行 Taylor 展开,并将 $k_1, k_2$ 代入 $\varphi$,要求 $\varphi$ 的前三项与公式 \ref{taylor} 中的增量函数相等,即可解得 $a, b, c$ (详见课本 P361)
			
		\end{subsubsection}

		\begin{subsubsection}{古典四级四阶 Runge-Kutta 方法}
		
			将 RK 方法用 RK 表描述 $$ \begin{array}{c|c}{c} & {A} \\ \hline & {b^{\top}}\end{array} $$ 
			
			古典四级四阶 Runge-Kutta 方法对应的 RK 表为 $$
			\begin{array}{c|cccc}{0} & {} & {} & {} & {} \\ {\frac{1}{2}} & {\frac{1}{2}} & {} & {} & {} \\ 
				{\frac{1}{2}} & {0} & {\frac{1}{2}} & {} & {} \\ 
				{1} & {0} & {0} & {1} & {} \\ \hline 
				{} & {1\over 6} & {1\over 3} & {1\over 3} & {1\over 6} \end{array}
			$$ 具体计算格式 $$
			\left\{\begin{array}{l}{y_{n+1}=y_{n}+\frac{1}{6} h\left(k_{1}+2 k_{2}+2 k_{3}+k_{4}\right)} \\ {k_{1}=f\left(x_{n}, y_{n}\right)} \\ {k_{2}=f\left(x_{n}+\frac{1}{2} h, y_{n}+\frac{1}{2} h k_{1}\right)} \\ {k_{3}=f\left(x_{n}+\frac{h}{2}, y_{n}+\frac{1}{2} h k_{2}\right)} \\ {k_{4}=f\left(x_{n}+h, y_{n}+h k_{3}\right)}\end{array}\right. $$
			
		\end{subsubsection}
		
	\end{subsection}
	
	\begin{subsection}{算法实现}
		
		古典四级四阶 Runge-Kutta 方法的 MATLAB 实现如下:
		
		\begin{lstlisting}[language=Matlab]
function y = myRungeKutta44(f, y0, a, b, h)
% 古典四级四阶 Runge-Kutta 方法
% 求解微分方程 dy/dx = f(x, y), x in [a,b]; y(a) = y0
% h 为步长
n = floor((b-a)/h);
x = a + h*(0:n); 
y = y0; 
yn = y0;
for i = 1:n
    xn = x(i);
    k1 = f(xn, yn);
    k2 = f(xn + h/2, yn + h/2*k1);
    k3 = f(xn + h/2, yn + h/2*k2);
    k4 = f(xn + h, yn + h*k3);
    yn1 = yn + h/6*(k1+2*k2+2*k3+k4);
    y = [y yn1]; yn = yn1;
end
end
		\end{lstlisting}
		
	\end{subsection}
	
\end{section}

\begin{section}{隐式二级四阶 Runge-Kutta 方法(Gauss 方法)}

	\begin{subsection}{算法原理}
		
		\begin{subsubsection}{隐式二级四阶 Runge-Kutta 方法}
			
			隐式二级四阶 Runge-Kutta 方法对应的 RK 表为 $$\begin{array}{c|cc}{\frac{1}{2}-\frac{\sqrt{3}}{6}} & {\frac{1}{4}} & {\frac{1}{4}-\frac{\sqrt{3}}{6}} \\ {\frac{1}{2}+\frac{\sqrt{3}}{6}} & {\frac{1}{4}+\frac{\sqrt{3}}{6}} & {\frac{1}{4}} \\ \hline & {\frac{1}{2}} & {\frac{1}{2}}\end{array}$$ 具体计算格式 $$\begin{array}{l}{y_{n+1}=y_{n}+h\left(\frac{1}{2} k_{1}+\frac{1}{2} k_{2}\right)} \\ {k_{1}=f\left(x_{n}+\left(\frac{1}{2}-\frac{\sqrt{3}}{6}\right) h, y_{n}+\frac{1}{4} h k_{1}+\left(\frac{1}{4}-\frac{\sqrt{3}}{6}\right) h k_{2}\right)} \\ {k_{2}=f\left(x_{n}+\left(\frac{1}{2}+\frac{\sqrt{3}}{6}\right) h, y_{n}+\left(\frac{1}{4}+\frac{\sqrt{3}}{6}\right) h k_{1}+\frac{1}{4} h k_{2}\right)}\end{array}$$
			
			
		\end{subsubsection}

		\begin{subsubsection}{隐式方法的迭代计算}
			
			可用迭代的方法求解隐式方法,具体地,先给出 $k_1, k_2$ 的近似值 $k_1^{(0)}, k_2^{(0)}$,然后用显式迭代 $$\begin{array}{l}{k_{1}^{(s+1)}=f\left(x_{n}+\left(\frac{1}{2}-\frac{\sqrt{3}}{6}\right) h, y_{n}+\frac{1}{4} h k_{1}^{(s)}+\left(\frac{1}{4}-\frac{\sqrt{3}}{6}\right) h k_{2}^{(s)}\right)} \\ {k_{2}^{(s+1)}=f\left(x_{n}+\left(\frac{1}{2}+\frac{\sqrt{3}}{6}\right) h, y_{n}+\left(\frac{1}{4}+\frac{\sqrt{3}}{6}\right) h k_{1}^{(s)}+\frac{1}{4} h k_{2}^{(s)}\right)}\end{array}$$ 直至 $||k_1^{(s)} -k_1^{(s+1)}||<\varepsilon, ||k_2^{(s)} -k_2^{(s+1)}||<\varepsilon$
			
			该方法可由 Gauss 求积公式导出,因此也称 Gauss 方法。
			
		\end{subsubsection}
		
	\end{subsection}
	
	\begin{subsection}{算法实现}
		
		隐式二级四阶 Runge-Kutta 方法的 MATLAB 实现如下:
		
		\begin{lstlisting}[language=Matlab]
function y = myRungeKutta24(f, y0, a, b, h, eps)
% 隐式二级四阶 Runge-Kutta 方法
% 求解微分方程 dy/dx = f(x, y), x in [a,b]; y(a) = y0
% h 为步长
c1 = 1/2-sqrt(3)/6; c2 = 1/2+sqrt(3)/6; 
b1 = 1/2; b2 = 1/2;
a11 = 1/4; a12 = 1/4-sqrt(3)/6;
a21 = 1/4+sqrt(3)/6; a22 = 1/4;

n = floor((b-a)/h);
x = a + h*(0:n); 
y = y0; 
yn = y0;
for i = 1:n
    xn = x(i);
    k1 = f(xn, yn); k2 = k1;
    while true % 迭代
        new_k1 = f(xn+c1*h, yn + a11*h*k1 + a12*h*k2);
        new_k2 = f(xn+c2*h, yn + a21*h*k1 + a22*h*k2);
        if (max_error(k1, new_k1)< eps && max_error(k2, new_k2)< eps)
            break
        end
        k1 = new_k1; k2 = new_k2;
    end
    yn1 = yn + h*(b1*k1+b2*k2);
    y = [y yn1]; yn = yn1;
end
end
		\end{lstlisting}
		
	\end{subsection}
	
\end{section}

\begin{section}{计算结果与方法比较}
	
	\begin{subsection}{误差}
	
		选取步长 $h=0.001$ ,隐式二级四阶 RK 方法中的 $eps = 10^{-7}$,以上两种方法的计算结果和误差见表 \ref{error_44} 和表 \ref{error_24}。
		
		可以看到两种方法都能得到较精确的数值解,但精度有差异,隐式二级四阶 RK 方法比古典四级四阶 RK 方法精度好。
		
	\end{subsection}

	\begin{subsection}{方法的阶和步长的影响}
		
		以上实现的两种方法都是4阶方法,取隐式二级四阶 RK 方法中的 $eps = 10^{-12}$,修改不同的步长 $h$ 得到计算结果如表 \ref{error_h}。
		
		其中平均误差定义为 $\text{mean}\{|y_n - y(x_n)|\}, {n = 0,1,\cdots}$,最大误差定义为 $\max\{|y_n - y(x_n)|\}, {n = 0,1,\cdots}$
		
		可以看到,随着 $h$ 减少 1 个数量级,误差的大小减少约 4 个数量级,这符合方法是 4 阶的。
		
	\end{subsection}
	
\end{section}
	
\begin{section}{总结}

	本次实验对古典四级四阶 RK 方法和隐式二级四阶 RK 方法进行了理论分析和编程计算,得到了一常微分方程初值问题的数值解,并比较了他们的计算误差。
	
	本次实验还探究了步长 $h$ 对误差的影响,结果符合预期,与两种方法为四阶方法的事实相符。
	
\end{section}

\end{multicols}

\begin{table*}[ht]
	\centering
	\caption{古典四级四阶 Runge-Kutta 方法的计算结果和误差}
	\label{error_44}
	\begin{tabular}{c|c|cc}
	\hline
	$x_n$ & $y(x_n)$           & $y_n$ (RK44)       & $y(x_n)-y_n $ (误差)   \\ \hline
	5     & (-0.4967, -1.9938) & (-0.4907, -1.9938) & $(6.0163\times 10^{-3}, -3.0089\times 10^{-6})$\\
	10    & (-0.4931, -1.9863) & (-0.4931, -1.9863) & $(2.5503\times 10^{-5}, -1.2755\times 10^{-8})$\\
	15    & (-0.4894, -1.9788) & (-0.4894, -1.9788) & $(1.0525\times 10^{-7}, -5.2636\times 10^{-11})$\\
	20    & (-0.4857, -1.9714) & (-0.4857, -1.9714) & $(4.3420\times 10^{-10}, -2.1672\times 10^{-13})$
	\\ \hline
	\end{tabular}
\end{table*}

\begin{table*}[ht]
	\centering
	\caption{隐式二级四阶 Runge-Kutta 方法的计算结果和误差}
	\label{error_24}
	\begin{tabular}{c|c|cc}
	\hline
	$x_n$ & $y(x_n)$           & $y_n$ (RK24)       & $y(x_n)-y_n $ (误差)                                \\ \hline
	5     & (-0.4967, -1.9938) & (-0.4967, -1.9938) & $(4.0484\times 10^{-5}, -2.0247\times 10^{-8})$  \\
	10    & (-0.4931, -1.9863) & (-0.4931, -1.9863) & $(4.7797\times 10^{-9}, -2.3901\times 10^{-12})$ \\
	15    & (-0.4894, -1.9788) & (-0.4894, -1.9788) & $(3.1969\times 10^{-12}, 1.3989\times 10^{-14})$ \\
	20    & (-0.4857, -1.9714) & (-0.4857, -1.9714) & $(1.2546\times 10^{-14}, 5.4401\times 10^{-14})$ \\ \hline
	\end{tabular}
\end{table*}

\begin{table*}[ht]
	\centering
	\caption{步长 $h$ 对误差的影响}
	\label{error_h}
	\begin{tabular}{c|cc|cc}
	\hline
	$h$       & 平均误差(RK44)                & 最大误差(RK44)                & 平均误差(RK24)                & 最大误差(RK24)                \\ \hline
	$10^{-3}$ & $4.300212\times 10^{-6}$  & $9.909147\times 10^{-2}$  & $1.367054\times 10^{-7}$  & $3.763211\times 10^{-3}$  \\
	$10^{-4}$ & $9.826336\times 10^{-11}$ & $2.900773\times 10^{-6}$  & $1.395697\times 10^{-11}$ & $4.100364\times 10^{-7}$  \\
	$10^{-5}$ & $1.551279\times 10^{-14}$ & $2.495640\times 10^{-10}$ & $8.662294\times 10^{-14}$ & $4.090728\times 10^{-11}$ \\ \hline
	\end{tabular}
\end{table*}

%\bibliographystyle{unsrt}
%\bibliography{ref.bib}

%\begin{thebibliography}{99}    %参考文献开始
%	\bibitem{ml}周志华. 机器学习[M]. 清华大学出版社, 2016.   
%\end{thebibliography}	

\end{document}

